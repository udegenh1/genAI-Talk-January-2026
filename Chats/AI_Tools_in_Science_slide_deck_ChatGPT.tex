\documentclass{beamer}
%
\usetheme[block=fill, background=light] {metropolis}
\setbeamertemplate{theorems}[]
\usepackage{FiraSans}
\usepackage{import}
\usepackage{xifthen}
\usepackage{pdfpages}
\usepackage{transparent}
\usepackage{float}
\usepackage[ngerman]{babel}
\usepackage{manfnt}
\usepackage{graphicx,import}
\usepackage{calc}
\usepackage{booktabs}
\usepackage{csquotes}
\usepackage{listings}
% Setup style for code sections
\definecolor{codegreen}{rgb}{0,0.6,0}
\definecolor{codegray}{rgb}{0.5,0.5,0.5}
\definecolor{codepurple}{rgb}{0.58,0,0.82}
\definecolor{backcolour}{rgb}{0.95,0.95,0.92}


% Setup style for code sections
\definecolor{codegreen}{rgb}{0,0.6,0}
\definecolor{codegray}{rgb}{0.5,0.5,0.5}
\definecolor{codepurple}{rgb}{0.58,0,0.82}
\definecolor{backcolour}{rgb}{0.95,0.95,0.92}


\lstdefinestyle{UDstyle}{
	backgroundcolor=\color{backcolour},   
	commentstyle=\color{codegreen},
	keywordstyle=\color{magenta},
	numberstyle=\tiny\color{codegray},
	stringstyle=\color{codepurple},
	basicstyle=\ttfamily\footnotesize,
	breakatwhitespace=false,         
	breaklines=true,                 
	captionpos=b,                    
	keepspaces=true,                 
	numbers=left,                    
	numbersep=5pt,                  
	showspaces=false,                
	showstringspaces=false,
	showtabs=false,                  
	tabsize=2
}

\lstset{style=UDstyle}


%{\tiny }
\usepackage{tabto}
\def\margindbend{\protect\marginsymbolhelper}
\def\marginsymbolhelper{\tabto*{-0.51cm}\makebox[0cm]{\dbend}\tabto*{\TabPrevPos}}

\bibliographystyle{mpidsrr_bibtexstyle}
%\usepackage{biblatex}
%\addbibresource{/Users/udegenh1/ownCloud/MPIDS/Forschung/UD_Literatur.bib}

\title{Large Language Models and Information Discovery}
%\title{Generative Sprachmodelle und das Orakel-Problem}
%\date{\today}
\date{03/05/2025}
\author{Ulrich Degenhardt\inst{1}}
%\institute{Max-Planck-Institute for  Dynamics and  Self-Organization}
\institute[MPIDS] {\inst{1}%
	Max-Planck-Institut for  Dynamics and  Self-Organization
	\newline Göttingen}




% ==================================================
% Talk: AI Tools in Science - Large Language Models in Scientific Workflows
% ==================================================

\title[AI Tools in Science]{AI Tools in Science:\\Large Language Models in Scientific Workflows}
\author[U.~Degenhardt]{Ulrich Degenhardt}
\date[40 min]{Draft Talk -- 40 minutes}

\begin{document}

% ----------------------------------
\begin{frame}
  \titlepage
  \begin{flushright}
    \footnotesize Time: 0:00--0:30
  \end{flushright}
\end{frame}

% ----------------------------------
\begin{frame}{Why This Talk?}
  \begin{itemize}
    \item Scientists quietly use LLMs in their workflows.
    \item But LLMs are known for hallucinations and unreliability.
    \item Goal: Explore what LLMs can \emph{actually} do in scientific workflows.
    \item Perspective: personal, experience-based, not systematic research.
  \end{itemize}
  \vfill
  \begin{flushright}
    \footnotesize Time: 0:30--2:00
  \end{flushright}
\end{frame}

% ----------------------------------
\begin{frame}{Example 1: The ``Magic/Miracle'' Cartoon}
  \begin{itemize}
    \item Vague memory: cartoon with scientists, equations, and ``magic/miracle'' in the middle.
    \item Google search fails: no precise keywords.
    \item LLM reconstructs the reference and finds Sidney Harris's famous cartoon.
    \item Demonstrates fuzzy retrieval and context reconstruction.
  \end{itemize}
  \vfill
  \begin{flushright}
    \footnotesize Time: 2:00--3:00
  \end{flushright}
\end{frame}

% ----------------------------------
\begin{frame}{Example 2: Solving a Math Problem from a Screenshot}
  \begin{itemize}
    \item Screenshot of a linear algebra exercise in a PDF.
    \item LLM reads and interprets the mathematics (including \LaTeX{} notation).
    \item Produces a full, seemingly correct proof.
    \item Key point: impressive reasoning behaviour --- but correctness must be checked.
  \end{itemize}
  \vfill
  \begin{flushright}
    \footnotesize Time: 3:00--4:00
  \end{flushright}
\end{frame}

% ----------------------------------
\begin{frame}{Early Lessons}
  \begin{itemize}
    \item LLMs can do tasks that go beyond simple search.
    \item They appear to reason, summarise, and infer.
    \item At the same time: hallucinations are frequent.
    \item Takeaway: impressive \emph{capabilities} but fragile \emph{reliability}.
  \end{itemize}
  \vfill
  \begin{flushright}
    \footnotesize Time: 4:00--7:00
  \end{flushright}
\end{frame}

% ----------------------------------
\begin{frame}{A Physicist's Skepticism About ML}
  \begin{itemize}
    \item From a physics viewpoint, ML often looks like extreme curve fitting.
    \item Neural networks with $\sim 10^{11}$ parameters: ``10th-order polynomial through 3 data points''.
    \item Lack of conceptual simplicity and universality.
    \item ML/AI often feels epistemically unsatisfying compared to physical theories.
  \end{itemize}
  \vfill
  \begin{flushright}
    \footnotesize Time: 7:00--8:30
  \end{flushright}
\end{frame}

% ----------------------------------
\begin{frame}{But It Works\ldots}
  \begin{itemize}
    \item Despite theoretical unease, ML works remarkably well in practice.
    \item Emergent behaviour at scale that no one expected 20 years ago.
    \item Physicists must grapple with tools that are powerful but opaque.
    \item This tension underlies the rest of the talk.
  \end{itemize}
  \vfill
  \begin{flushright}
    \footnotesize Time: 8:30--10:00
  \end{flushright}
\end{frame}

% ----------------------------------
\begin{frame}{Practical Realities of LLMs}
  \begin{itemize}
    \item Stochastic: same question can yield different answers.
    \item Very verbose: Brandolini's Law --- debunking $>$ generating.
    \item Limited context window: they \emph{forget} earlier parts of the conversation.
    \item Golden rule: ask questions that can be verified or falsified.
  \end{itemize}
  \vfill
  \begin{flushright}
    \footnotesize Time: 10:00--11:30
  \end{flushright}
\end{frame}

% ----------------------------------
\begin{frame}{Infrastructure for Scientific Use}
  \begin{itemize}
    \item Treat LLM interactions like experiments.
    \item Use note-taking tools (e.g.\ Obsidian) as a ``second brain''.
    \item Export chats as markdown and store them as a lab notebook for AI.
    \item This makes results reproducible and reviewable later.
  \end{itemize}
  \vfill
  \begin{flushright}
    \footnotesize Time: 11:30--13:00
  \end{flushright}
\end{frame}

% ----------------------------------
\begin{frame}{Behaviour Control}
  \begin{itemize}
    \item Prompting can change global behaviour across a session.
    \item Examples:
      \begin{itemize}
        \item Numbered dialogue format for structured discussion.
        \item ``No Yes-Man'': explicitly ask the model to challenge weak ideas.
        \item \emph{Advocatus diaboli}: temporarily activate an adversarial role.
      \end{itemize}
    \item Aim: steer LLMs toward scientifically useful and critical behaviour.
  \end{itemize}
  \vfill
  \begin{flushright}
    \footnotesize Time: 13:00--16:00
  \end{flushright}
\end{frame}

% ----------------------------------
\begin{frame}{LLMs and Anthropomorphism}
  \begin{itemize}
    \item LLMs are conversationally fluent: they effectively pass the Turing test.
    \item Humans are primed to attribute mind, emotion, and intention.
    \item But LLMs have no consciousness, goals, or feelings.
    \item Empathy, politeness, humour: all simulated pattern generation.
  \end{itemize}
  \vfill
  \begin{flushright}
    \footnotesize Time: 16:00--17:30
  \end{flushright}
\end{frame}

% ----------------------------------
\begin{frame}{System Trust vs.\ Social Trust}
  \begin{itemize}
    \item Appropriate analogy: tools like \texttt{sed}, Linux, LAPACK.
    \item These tools are ``battle-tested'' in adversarial, high-pressure use.
    \item Trust is based on documented behaviour and known failure modes.
    \item LLMs should be evaluated as technical instruments, not colleagues.
  \end{itemize}
  \vfill
  \begin{flushright}
    \footnotesize Time: 17:30--21:00
  \end{flushright}
\end{frame}

% ----------------------------------
\begin{frame}{Stage 1: ``Google on Steroids''}
  \begin{itemize}
    \item Paste compiler or interpreter error messages into the LLM.
    \item Get direct explanations and potential fixes.
    \item Ask ``manual-level'' questions (e.g.\ matplotlib legends, API usage).
    \item Saves time compared to manual search + StackOverflow.
  \end{itemize}
  \vfill
  \begin{flushright}
    \footnotesize Time: 21:00--22:30
  \end{flushright}
\end{frame}

% ----------------------------------
\begin{frame}{Stage 2: Instant Prototyping}
  \begin{itemize}
    \item Turn ideas into minimal working examples quickly.
    \item Example: path signatures from CSV time series with \texttt{iisignature}.
    \item LLMs can generate drafts in many languages: Python, Julia, C++, Fortran, Go.
    \item Barrier from ``I have an idea'' to ``I have running code'' becomes tiny.
  \end{itemize}
  \vfill
  \begin{flushright}
    \footnotesize Time: 22:30--24:00
  \end{flushright}
\end{frame}

% ----------------------------------
\begin{frame}{Stage 3: Development Assistance}
  \begin{itemize}
    \item Example project: Cryptomator-like encryption in Go.
    \item Long initial prompt describing requirements and constraints.
    \item LLM proposes architecture, cryptographic choices, and commands.
    \item Then an iterative loop: compile, copy errors, let LLM fix, repeat.
  \end{itemize}
  \vfill
  \begin{flushright}
    \footnotesize Time: 24:00--27:00
  \end{flushright}
\end{frame}

% ----------------------------------
\begin{frame}{Leverage and Risks}
  \begin{itemize}
    \item Feels like having a bright, tireless PhD student as an assistant.
    \item Greatly increases personal leverage in software development.
    \item But: easy to become lazy and stop checking changes carefully.
    \item Security and correctness require disciplined review --- AI amplifies both good and bad.
  \end{itemize}
  \vfill
  \begin{flushright}
    \footnotesize Time: 27:00--29:00
  \end{flushright}
\end{frame}

% ----------------------------------
\begin{frame}{AI for Literature Search}
  \begin{itemize}
    \item Information abundance: hundreds of millions of papers.
    \item Semantic Scholar as a large bibliographic database.
    \item Tools like scienceOS build on top of it for scientists.
    \item Much better results for literature search than general-purpose LLMs.
  \end{itemize}
  \vfill
  \begin{flushright}
    \footnotesize Time: 29:00--30:30
  \end{flushright}
\end{frame}

% ----------------------------------
\begin{frame}{Working with Semantic Scholar Tools}
  \begin{itemize}
    \item Example query: ``All publications of Ramin Golestanian from 2025''.
    \item Tool returns a curated list with short descriptions.
    \item Citation network visualisation helps see connections.
    \item Limitation: typically up to $\sim 100$ sources per query.
  \end{itemize}
  \vfill
  \begin{flushright}
    \footnotesize Time: 30:30--32:00
  \end{flushright}
\end{frame}

% ----------------------------------
\begin{frame}{Document Analysis and Evidence Tracking}
  \begin{itemize}
    \item Upload many PDFs (100+) and ask questions across the whole set.
    \item Answers draw on both Semantic Scholar metadata and full text.
    \item scienceOS highlights exact locations in PDFs that support each answer.
    \item This explicit grounding increases scientific trust in the tool's output.
  \end{itemize}
  \vfill
  \begin{flushright}
    \footnotesize Time: 32:00--35:00
  \end{flushright}
\end{frame}

% ----------------------------------
\begin{frame}{Why LLM Summaries Are Hard to Remember}
  \begin{itemize}
    \item Generating summaries is easy; remembering them is not.
    \item Cognitive psychology: memory needs encoding, storage, and retrieval.
    \item Passive storage in Obsidian is not enough for long-term learning.
    \item Need spacing, retrieval practice, and active processing.
  \end{itemize}
  \vfill
  \begin{flushright}
    \footnotesize Time: 35:00--36:30
  \end{flushright}
\end{frame}

% ----------------------------------
\begin{frame}{A Practical Workflow for Summaries}
  \begin{itemize}
    \item Step 1: Ask LLM for a detailed summary of a paper.
    \item Step 2: Write your own 3-sentence meta-summary:
      \begin{enumerate}
        \item What is this about?
        \item What is the core idea?
        \item Why is this important or surprising?
      \end{enumerate}
    \item Step 3: Ask LLM to generate a concept map with labelled edges.
    \item Combine text + visual map in your notes for better retrieval.
  \end{itemize}
  \vfill
  \begin{flushright}
    \footnotesize Time: 36:30--39:00
  \end{flushright}
\end{frame}

% ----------------------------------
\begin{frame}{Final Reflections}
  \begin{itemize}
    \item Hallucinations are not unique to LLMs; humans make mistakes too.
    \item Safety comes from critical thinking, scientific method, and engineering practice.
    \item LLM suggestions are starting points, not final answers.
    \item Open questions:
      \begin{itemize}
        \item Better evidence for behaviour-control strategies.
        \item Using LLMs as intellectual sparring partners.
        \item Understanding GitHub repositories with AI.
        \item Richer descriptions of AI-assisted development workflows.
      \end{itemize}
  \end{itemize}
  \vfill
  \begin{flushright}
    \footnotesize Time: 39:00--40:00
  \end{flushright}
\end{frame}

\end{document}
